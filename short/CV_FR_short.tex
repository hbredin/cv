\documentclass{simplecv}
\usepackage[margin=1.3in]{geometry}


\begin{document}
\pagestyle{empty} 

\leftheader{LIMSI-CNRS\\BP 133\\91403 Orsay Cedex\\France
}

\rightheader{TEL: +33 (0) 1 69 85 81 84\\
\texttt{\small bredin@limsi.fr}\\
\texttt{\small http://herve.niderb.fr/}
}

\title{Herv\'{e} Bredin -- Charg\'{e} de Recherche CNRS}

\maketitle

\vspace{-.5cm}
\begin{center}
(derni\`{e}re mise \`{a} jour le 22 juillet 2012)
\end{center}

\section{Exp\'{e}riences professionnelles}

\begin{topic}
\item[2008 -- aujourd'hui] Charg\'{e} de Recherche au Centre National de la Recherche Scientifique (CNRS)

\emph{Fusion multimodale \& ses applications.} 

\begin{topic}
	\item[2010 -- aujourd'hui] Laboratoire d'Informatique pour la M\'{e}canique et les Sciences de l'Ing\'{e}nieur (Orsay, France)
	\item[2008 -- 2010] Institut de Recherche en Informatique de Toulouse (Toulouse, France)
\end{topic}

Mes travaux de recherche portent sur l'indexation des vid\'{e}os par leur contenu~\cite{Bredin2012a}, la structuration et le r\'{e}sum\'{e} des documents audiovisuels~\cite{Bredin2012}, ainsi que la reconnaissance multimodale des personnes.

\textbf{Projets :} Quaero, QCompere/REPERE, IRIM/TRECVid

\item[2008 (janvier -- septembre)] Chercheur post-doctoral \`{a} Dublin City University,\\ Center for Digital Video Processing (Dublin, Ireland)

\emph{R\'{e}sum\'{e} automatique de contenu vid\'{e}o produit par les utilisateurs.}

\item[2006 -- 2007] Charg\'{e} d'enseignement/recherche \`{a} T\'{e}l\'{e}com ParisTech (Paris, France)

\emph{Reconnaissance des formes et biom\'{e}trie.}

\item[2004 -- 2007] Pr\'{e}paration d'une th\`{e}se de doctorat \`{a} T\'{e}l\'{e}com ParisTech (Paris, France).

\emph{V\'{e}rification biom\'{e}trique de l'identit\`{e} par visage parlant.}

\textbf{Projets :} BioSecure, SecurePhone 

\item[2004 (juin -- d\'{e}cembre)] Stage de recherche au sein du groupe \emph{Parole} d'IBM (La D\'{e}fense, France)

\emph{Adaptation du syst\`{e}me de pr\'{e}-traitement phon\'{e}tique d'un syst\`{e}me de synth\`{e}se de parole fran\c{c}ais-canadien.}
\end{topic}

\section{Dipl\^{o}mes}

\begin{topic}
\item[2007] Th\`{e}se de doctorat d\'{e}livr\'{e}e par T\'{e}l\'{e}com ParisTech (Paris, France)

\emph{V{\'e}rification de l'identit{\'e} d'un visage parlant. Apport de la mesure de synchronie audiovisuelle face aux tentatives d{\'e}lib{\'e}r{\'e}es d'imposture.}

\item[2004] Dipl\^{o}me d'ing\'{e}nieur d\'{e}livr\'{e} par T\'{e}l\'{e}com ParisTech (Paris, France)

\emph{Reconnaissance des formes, traitement du signal et des images.}
\end{topic}

\section{Logiciels libres}
\begin{topic}
\item[PyAnnote] Module Python pour l'annotation collaborative de contenus multim\'{e}dia\\
\texttt{\small http://packages.python.org/PyAnnote}
\item[pinocchIO] Format de fichier, biblioth\`{e}que C et outils pour le stockage de descripteurs de documents audiovisuels\\
\texttt{\small http://pinocchio.niderb.fr}
\item[PyAFE] Outil Python pour l'\'{e}valuation des techniques d'\emph{audio fingerprinting}~\cite{Ramona2011}\\
\texttt{\small http://pyafe.niderb.fr}
\item[BioSecure] Syst\`{e}me de r\'{e}f\'{e}rence pour la reconnaissance des visages parlants~\cite{Bredin2006a}\\
\texttt{\small http://biosecure.it-sudparis.eu}
\end{topic}

\section{Publications \& prix}

5 articles de journaux, 3 chapitres d'ouvrage, 22 papiers de conf\'{e}rence.\\
\emph{EBF European Biometrics Research Industry Award 2007}~\cite{Bredin2008}

\section{S\'{e}lection de publications}
\nocite{Bredin2012,Bredin2012a,Ramona2011,Bredin2009,Bredin2008,Bredin2007,Argones-Rua2007a,Bredin2006a}
\bibliographystyle{abbrv}
\bibliography{publi/bredin}

\end{document}
